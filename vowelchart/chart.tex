\documentclass[12pt]{standalone}

\usepackage{tikz}
\usetikzlibrary{arrows.meta}
% Requires xelatex for the magnificent Brill font
\usepackage{libertine}

\begin{document}
	\begin{tikzpicture}[scale=3,baseline=default]
	%modified from http://tex.stackexchange.com/a/171541
	\large
	\tikzset{
		vowel/.style={fill=white, anchor=mid, text depth=0ex, text height=1ex},
		dot/.style={circle,fill=black,minimum size=0.4ex,inner sep=0pt,outer sep=-1pt},
	}
	\coordinate (hf) at (0,2); % high front
	\coordinate (hb) at (2,2); % high back
	\coordinate (lf) at (1,0); % low front
	\coordinate (lb) at (2,0); % low back
	\def\V(#1,#2){barycentric cs:hf={(3-#1)*(2-#2)},hb={(3-#1)*#2},lf={#1*(2-#2)},lb={#1*#2}}
	
	% Draw the chart key (vorne -- hinten)
	\draw [{Latex[round]}-] (\V (-.25,0)) -- (\V (-.25,.5)) node[above left] {\footnotesize vorne};
	\draw [-{Latex[round]}] (\V (-.25,1.5)) -- (\V (-.25,2)) node[above left] {\footnotesize hinten};
	\path (\V (-.25,1)) node[above] {\footnotesize zentral};
	
	% hoch--tief
	\draw [{Latex[round]}-] (\V (0,-.25)) -- +(270:.5cm) node[above right,rotate=90] (vokaltrapez1) {\footnotesize hoch};
	\draw [{Latex[round]}-] (\V (3,-2.5)) -- +(270:-.5cm) node[above left,rotate=90] (vokaltrapez2) {\footnotesize tief};
	\path (\V (1.5,-1)) node[above,rotate=90] {\footnotesize mittel};

	% Draw the horizontal lines first.
	\draw [gray] (\V(0,0)) -- (\V(0,2));
	\draw [gray] (\V(1,0)) -- (\V(1,2));
	\draw [gray] (\V(2,0)) -- (\V(2,2));
	\draw [gray] (\V(3,0)) -- (\V(3,2));
	
	% Draw the vertical lines.
	\draw [gray] (\V(0,0)) -- (\V(3,0));
	\draw [gray] (\V(0,1)) -- (\V(3,1));
	\draw [gray] (\V(0,2)) -- (\V(3,2));
	
	% Draw a universal border
	\draw [gray] (\V(-.15,-.2)) -- (\V(3.15,-.75)) -- (\V(3.15,2.25)) -- +(270:-2.2cm) -- (\V(-.15,-.2));
	
	% Place all the unrounded-rounded pairs next, on top of the horizontal lines.
	\path (\V(0,0))     node[vowel, left] {i} node[vowel, right] (y) {y} node[dot] {};
%	\path (\V(0,1))     node[vowel, left] {ɨ} node[vowel, right] {ʉ} node[dot] {};
%	\path (\V(0,2))     node[vowel, left] {ɯ} node[vowel, right] {u} node[dot] {};
	\path (\V(0.5,0.4)) node[vowel, left] {ɪ} node[vowel, right] (Y) {ʏ} node[dot] {};
	\path (\V(0.5,1.6)) node[vowel, left] { } node[vowel, right] (uu) {ʊ} node[dot] {};
	\path (\V(1,0))     node[vowel, left] {e} node[vowel, right] (e) {ø} node[dot] {};
%	\path (\V(1,1))     node[vowel, left] {ɘ} node[vowel, right] {ɵ} node[dot] {};
%	\path (\V(1,2))     node[vowel, left] {ɤ} node[vowel, right] {o} node[dot] {};
	\path (\V(2,0))     node[vowel, left] {ɛ} node[vowel, right] (ee) {œ} node[dot] {};
%	\path (\V(2,1))     node[vowel, left] {ɜ} node[vowel, right] {ɞ} node[dot] {};
%	\path (\V(2,2))     node[vowel, left] {ʌ} node[vowel, right] {ɔ} node[dot] {};
%	\path (\V(2.5,0))   node[vowel, left] {æ} node[vowel, right] { } node[   ] {};
%	\path (\V(3,0))     node[vowel, left] {a} node[vowel, right] {} node[dot] {};
%	\path (\V(3,2))     node[vowel, left] {ɑ} node[vowel, right] {ɒ} node[dot] {};
	
	
	% Place the unpaired symbols last, on top of the vertical lines.
	\path (\V(1.5,1))   node[vowel]   (schwa)   {ə};
	\path (\V(2.5,1))   node[vowel]   (schwaa)  {ɐ};
	\path (\V(3,0))		node[vowel]		(a)  	{a};
	\path (\V (2,2))	node[vowel]		(oo)  	{ɔ};
	\path (\V (1,2))	node[vowel]		 (o) 	{o};
	\path (\V (0,2))	node[vowel]		 (u) 	{u};
	
	% Draw connections
	\draw [-{Latex[round]}] (y) -- (schwaa);
	\draw [-{Latex[round]}] (Y) -- (schwa);
	\path (e) edge [-{Latex[round]},bend right=10] (schwaa);
	\path (ee) edge [-{Latex[round]},bend left=20] (schwa);
	\draw [-{Latex[round]}] (a) -- (schwaa);
	\draw [-{Latex[round]}] (oo) -- (schwa);
	\path (o) edge [-{Latex[round]},bend left=15] (schwaa);	
	\path (u) edge [-{Latex[round]},bend left=10] (schwaa);	
	\draw [-{Latex[round]},bend right] (uu) -- (schwa);
	\end{tikzpicture}
\end{document}