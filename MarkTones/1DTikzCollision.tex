\documentclass{article}
\usepackage{ifthen,libertine,tikz,pgfmath,calc,xparse}
\usetikzlibrary{calc,positioning,decorations.text}

\begin{document}
	\def\LSPAlignTonesLastnode{none}
	\def\none{none}
	\def\LSPAlignTonesXshift{0mm}
	\NewDocumentEnvironment{aligntones}{m}{%
	\begin{tikzpicture} [
	baseline, 
	aligntones/.style={
	anchor=base,
	decorate,
	decoration={text effects along path,
		text={##1}, % #1 == ##1
		text effects/.cd,
		path from text, character count=\i,
		characters={text along path,text depth=0ex,name=\i},
	    text effects={characters/.append={}}
    	},
	},
	belowcharnumber/.style={
		anchor=base,
		below=\baselineskip of ##1,
		name=##1i,
		xshift=\LSPAlignTonesXshift,
		append after command={
			\pgfextra{\draw (##1.south) -- (##1i);
			\xdef\LSPAlignTonesLastnode{##1i} % save the last node for further reference
			}
		}
	}
]
\path [aligntones=#1] (0,0);
}{\end{tikzpicture}}

\NewDocumentCommand{\AlignTones}{m}{
\foreach \a/\b in {#1} {
	\ifx\LSPAlignTonesLastnode\none % has there been a previous node?
		\node[belowcharnumber=\a,inner xsep=0pt,draw=blue] {\b};
	\else % There has not been a previous node
		\def\LSPAlignTonesNewnode{\a}
		\def\LSPAlignTonesNewentry{\b}
		\coordinate (A) at (\LSPAlignTonesLastnode);
		\path let \p1 = (A), \p2 = (\LSPAlignTonesNewnode) in node (B) at (\x2,\y1) [minimum width=2.5pt, minimum height=2ex,draw=red,inner xsep=0pt] {\settowidth{\dimen0}{\LSPAlignTonesNewentry}\hspace*{\dimen0}}; 
		\path let  \p1 = ( $(\LSPAlignTonesLastnode.east) - (B.west)$ ), \n1 = {veclen(\x1,\y1)} 
		in \pgfextra{
			\pgfmathanglebetweenpoints{\pgfpointanchor{\LSPAlignTonesLastnode}{east}}{\pgfpointanchor{B}{west}}
			\let\abangle\pgfmathresult
			\pgfmathparse{or(scalar(\n1)<2,notequal(scalar(\abangle),0))} % 2pt is the width of a small space
			\ifthenelse{\pgfmathresult=1}{\pgfmathsetlengthmacro{\LSPAlignTonesXshift}{.75em}}{\pgfmathsetlengthmacro{\LSPAlignTonesXshift}{0pt}}
		} node[belowcharnumber=\LSPAlignTonesNewnode,align=center] {\LSPAlignTonesNewentry};
	\fi
	}
}
\begin{aligntones}{My text} 
	\AlignTones{2/HL!L, 7/H}
\end{aligntones}


\end{document}