\documentclass{article}
\usepackage{ifthen,libertine}
\usepackage{tikz,pgfmath,calc,xparse}
\usetikzlibrary{calc,positioning,decorations.text}
\newcommand{\myshift}{}
\begin{document}
%	\NewDocumentCommand \AlignTones {m m} {
		
	\begin{tikzpicture} [aligntones/.style={decoration={text effects along path,
	text={#1},
	text effects/.cd,
	path from text, character count=\i,
	characters={text along path,text depth=0ex,name=\i}},
}]
	\path [decorate,aligntones=Text, text effects={characters/.append={}}] (0,0);	
	\foreach \x/\y in {#2} {
	\node[below=\baselineskip of \y] (\y i) {\x};
	\draw[solid] (\y) -- (\y i);};
		
%	\node [inner xsep=0pt,draw=red] (A) at (1,5) {Test};
%	\node [inner xsep=0pt,draw=green] (B) at (1.25,5) {Test2};
%	\path let  \p1 = ( $(A.east) - (B.west)$ ), \n1 = {veclen(\x1,\y1)} 
%	in \pgfextra{
%			\pgfmathanglebetweenpoints{\pgfpointanchor{A}{east}}{\pgfpointanchor{B}{west}}\let\abangle\pgfmathresult
%		    \pgfmathparse{or(scalar(\n1)<2.5,notequal(scalar(\abangle),0))} % 2.5pt is the width of a space in Libertine 12pt
%			\ifthenelse{\pgfmathresult=1}{\pgfmathsetlengthmacro{\myshift}{5em}}{\pgfmathsetlengthmacro{\myshift}{0pt}}
%		} node[below=of A.west,draw,xshift=\myshift,anchor=west] {\myshift \pgfmathparse{scalar(\n1)}\pgfmathresult \abangle};
	\end{tikzpicture}
%}

%\AlignTones{Nice Text!}{HL/1}
\end{document}